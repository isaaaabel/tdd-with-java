\chapter{Fazit}
Diese Studienarbeit zielt darauf ab, eine intuitive Benutzererfahrung zu bieten, die es Benutzern ermöglicht, Todos zu erstellen, zu bearbeiten und zu löschen, sowie zusätzliche Details wie Beschreibungen und Fälligkeitsdaten hinzuzufügen.

Die entwickelte Todo-Liste besteht aus zwei Hauptkomponenten: dem Frontend, das auf Vue.js basiert und die Benutzeroberfläche bereitstellt, und dem Backend, das auf Spring Boot aufbaut und die Datenverarbeitung und -speicherung übernimmt. Die Integration mit einer MySQL-Datenbank bietet eine robuste Grundlage für die Speicherung von Benutzerdaten und Todos.

Trotz der erfolgreichen Implementierung der grundlegenden Funktionen der Todo-Liste gibt es noch einige Herausforderungen und unvollständige Aspekte, die erwähnt werden müssen. Ein Benutzer kann sich erfolgreich registrieren und anmelden, jedoch sind die Funktionen zur sicheren Abmeldung und zur Bearbeitung von Todos nicht vollständig implementiert. Dies führt zu Bedenken hinsichtlich der Privatsphäre und der Benutzerfreundlichkeit der Anwendung.

Die Gründe für diese unvollständigen Funktionen sind vielschichtig. Der begrenzte Erfahrungsschatz des Entwicklers mit den verwendeten Technologien sowie unzureichendes Zeitmanagement haben dazu geführt, dass der ursprüngliche Zeitplan nicht eingehalten werden konnte. Zudem wurde der Zyklus des Test-Driven Developments (TDD) nicht konsequent durchgeführt vor allem im Hinblick des Refactoring, was zu nicht bestandenen Tests und vermutlich fehlerhaftem Code führte. Diese Probleme haben sich direkt auf die Funktionalität der Anwendung ausgewirkt, insbesondere auf Bereiche wie das sichere Ausloggen und die Bearbeitung von Todos.

Für zukünftige Weiterentwicklungen und Verbesserungen der Todo-Liste Webanwendung ist es entscheidend, diese Herausforderungen zu adressieren. Ein verstärkter Fokus auf Schulung und Weiterbildung in den verwendeten Technologien sowie eine verbesserte Projektplanung und -überwachung können helfen, ähnliche Probleme in zukünftigen Entwicklungsprojekten zu vermeiden. Darüber hinaus sollte die Implementierung von Best Practices im Bereich Softwareentwicklung, einschließlich eines strengeren TDD-Ansatzes und einer umfassenden Testabdeckung, Priorität haben, um die Qualität und Zuverlässigkeit der Anwendung sicherzustellen.

Insgesamt bietet diese Studienarbeit einen detaillierten Einblick in die Architektur, Implementierung und die Lessons Learned bei der Entwicklung der Todo-Liste Webanwendung. Trotz der identifizierten Herausforderungen legt sie den Grundstein für künftige Verbesserungen und Erweiterungen dieser digitalen Werkzeuge zur Aufgabenverwaltung.