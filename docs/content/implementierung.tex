\chapter{Implementierung unter Verwendung von TDD}

\section{Entwicklungsumgebung und Werkzeuge}

Die Entwicklung der Todo-App erfolgte in einer modernen Java-Entwicklungsumgebung, die auf dem Spring Boot Framework basiert. Die wichtigsten verwendeten Tools und Technologien sind:

\begin{itemize}
	\item \textbf{IDE}: IntelliJ IDEA wurde als Integrated Development Environment (IDE) verwendet, da es umfangreiche Unterstützung für Java und das Spring Framework bietet, einschließlich leistungsstarker Debugging- und Refactoring-Tools.
	\item \textbf{Build-Tool}: Maven wurde für das Build-Management und die Abhängigkeitsverwaltung eingesetzt. Es ermöglicht die einfache Verwaltung von Bibliotheken und Plugins sowie die Konfiguration von Build-Prozessen.
	\item \textbf{Versionierung}: Git und GitHub wurden für die Quellcodeverwaltung und Versionskontrolle verwendet. GitHub ermöglichte zudem die Zusammenarbeit und den Austausch von Code.
	\item \textbf{Test-Frameworks}: JUnit 5 und Spring Boot Test wurden für die Implementierung und Ausführung von Unit- und Integrationstests verwendet. Mockito diente zur Erstellung von Mock-Objekten für die Isolierung von Testfällen.
	\item \textbf{Datenbank}: MySQL wurde als relationale Datenbank verwendet. Für die Tests wurde eine MySQL-Testdatenbank konfiguriert, um schnelle und isolierte Testausführungen zu ermöglichen.
\end{itemize}

\section{Implementierungsschritte}

Die Implementierung der Benutzerregistrierung in der Todo-App folgte dem klassischen TDD-Zyklus: \textbf{Red-Green-Refactor}.

\begin{enumerate}
	\item \textbf{Red Phase}: Zunächst wurde ein fehlgeschlagener Test (Red) geschrieben, der die Registrierung eines neuen Benutzers beschreibt, ohne dass die Implementierung vorhanden war.
	
	Beispiel: Ein Test für die Registrierung eines neuen Benutzers über den UserController.
	\begin{lstlisting}[language=Java]
		@Test
		public void testRegisterUser_success() throws Exception {
			UserRegistrationRequest request = new UserRegistrationRequest();
			request.setUsername("integrationtestuser");
			request.setPassword("password");
			request.setConfirmPassword("password");
			
			ResultActions result = mockMvc.perform(post("/register")
			.contentType(MediaType.APPLICATION_JSON)
			.content(objectMapper.writeValueAsString(request)));
			
			result.andExpect(status().isOk());
		}
	\end{lstlisting}
	
	\item \textbf{Green Phase}: Anschließend wurde der minimal notwendige Code geschrieben, um den Test erfolgreich zu bestehen (Green). In diesem Schritt wird nur so viel implementiert, dass der Testfall erfüllt wird.
	
	Beispiel: Die grundlegende Implementierung des registerUser-Endpunkts im UserController.
	
	\begin{lstlisting}[language=Java]
		@PostMapping("/register")
		public ResponseEntity<?> registerUser(@RequestBody UserRegistrationRequest request) {
			User user = new User();
			user.setUsername(request.getUsername());
			user.setPassword(passwordEncoder.encode(request.getPassword()));
			
			return ResponseEntity.ok(user);
		}
	\end{lstlisting}
	
	\item \textbf{Refactor Phase}: Nach dem erfolgreichen Bestehen des Tests wurde der Code optimiert und verbessert, ohne die Funktionalität zu ändern. Dabei wurde auf Sauberkeit, Lesbarkeit und Wartbarkeit des Codes geachtet.
	
	Beispiel: Die vollständige Implementierung des registerUser-Endpunktes im UserController zeigt, dass ein Refactoring erfolgte, nachdem die Implementierung zum erfolgreichen Ausführen von Test, wie zum Beispiel des Testens der Fehlerbehandlung von bereits existierenden Benutzernamen, hinzugefügt wurde.
	
	\begin{lstlisting}[language=Java]
		@PostMapping("/register")
		public ResponseEntity<?> registerUser(@Valid @RequestBody UserRegistrationRequest request, BindingResult result) {
			// Check if username already exists
			if (userService.existsByUsername(request.getUsername())) {
				return ResponseEntity.status(HttpStatus.CONFLICT).body("Username already exists");
			}
			
			// Check if password and confirmation match
			if (!request.getPassword().equals(request.getConfirmPassword())) {
				
				return ResponseEntity.status(HttpStatus.CONFLICT).body("Password and confirm password do not match");
			}
			
			// Check for validation results
			if (result.hasErrors()) {
				
				return ResponseEntity.status(HttpStatus.BAD_REQUEST).body("Validation error: " + result.getAllErrors());
			}
			
			// Register user
			try {
				User user = authenticationService.register(request);
				return ResponseEntity.ok(user);
			} catch (Exception e) {
				return ResponseEntity.status(HttpStatus.INTERNAL_SERVER_ERROR).body("Error registering user");
			}
		}
	\end{lstlisting}
	
\end{enumerate}

Die konsequente Anwendung des TDD-Zyklus gewährleistet, dass die Implementierung kontinuierlich durch Tests begleitet und abgesichert wird und somit zu einem stabilen und fehlerarmen Code führen kann.

\section{Beispieltests und Testfälle}

\section{Unit-Tests}

Unit-Tests wurden verwendet, um einzelne Komponenten isoliert zu testen. Ein Beispiel ist der Test der \texttt{UserController}-Klasse, um sicherzustellen, dass eine Registrierung eines Benutzers erfolgreich verläuft.

\begin{lstlisting}[language=Java]
	@Test
	public void testRegisterUserPasswordMismatch() {
		UserRegistrationRequest request = new UserRegistrationRequest("testuser", "password1", "password2");
		
		BindingResult result = mock(BindingResult.class);
		ResponseEntity<?> response = userController.registerUser(request, result);
		
		assertEquals(HttpStatus.CONFLICT, response.getStatusCode());
		assertEquals("Password and confirm password do not match", response.getBody());
	}
\end{lstlisting}

\section{Integrationstests}

Integrationstests wurden durchgeführt, um das Zusammenspiel verschiedener Komponenten zu überprüfen. Ein Beispiel ist der Integrationstest für die Benutzerregistrierung über den \texttt{UserController}.

\begin{lstlisting}[language=Java]
	@SpringBootTest(webEnvironment = SpringBootTest.WebEnvironment.RANDOM_PORT)
	public class UserRegistrationAcceptanceTest {
		
		@LocalServerPort
		private int port;
		
		@Autowired
		private TestRestTemplate restTemplate;
		
		@Test
		public void testRegisterUser() {
			String url = "http://localhost:" + port + "/register";
			UserRegistrationRequest request = new UserRegistrationRequest("testuser", "password", "password");
			
			HttpHeaders headers = new HttpHeaders();
			HttpEntity<UserRegistrationRequest> entity = new HttpEntity<>(request, headers);
			
			// Send HTTP POST request
			ResponseEntity<String> response = restTemplate.exchange(url, HttpMethod.POST, entity, String.class);
			
			// Check if the response status code is 409 (CONFLICT)
			assertEquals(HttpStatus.CONFLICT, response.getStatusCode());
		}
	}
\end{lstlisting}


\section{Akzeptanztests}

Akzeptanztests wurden verwendet, um sicherzustellen, dass die Anwendung den Anforderungen der Benutzer entspricht. Diese Tests wurden aus der Sicht des Endbenutzers geschrieben und überprüfen die Funktionalität der gesamten Anwendung.

Ein Beispiel ist der Akzeptanztest für die Benutzerregistrierung über die REST-Schnittstelle.


\begin{lstlisting}[language=Java]
	@SpringBootTest(webEnvironment = SpringBootTest.WebEnvironment.RANDOM_PORT)
	public class UserRegistrationAcceptanceTest {
		
		@LocalServerPort
		private int port;
		
		@Autowired
		private TestRestTemplate restTemplate;
		
		@Test
		public void testRegisterUser() {
			String url = "http://localhost:" + port + "/register";
			UserRegistrationRequest request = new UserRegistrationRequest("testuser", "password", "password");
			
			HttpHeaders headers = new HttpHeaders();
			HttpEntity<UserRegistrationRequest> entity = new HttpEntity<>(request, headers);
			
			// Send HTTP POST request
			ResponseEntity<String> response = restTemplate.exchange(url, HttpMethod.POST, entity, String.class);
			
			// Check if the response status code is 409 (CONFLICT)
			assertEquals(HttpStatus.CONFLICT, response.getStatusCode());
		}
	}
\end{lstlisting}

Diese strukturierte Vorgehensweise durch TDD führte zu einer robusten und fehlerarmen Implementierung der Todo-App, da jedes Feature durch entsprechende Tests abgesichert und validiert wurde.

